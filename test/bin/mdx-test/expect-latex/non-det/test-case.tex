`ocaml-mdx` supports non-determinitic code blocks.

There are two kinds of blocks:

### Non-deterministic Outputs

Code blocks with `non-deterministic=output` have their command always
executed but their output is never checked, unless `--non-deterministic`
is passed as argument to `ocaml-mdx`.


% $MDX non-deterministic=output
\begin{sh}
$ echo $RANDOM
4150
\end{sh}

% $MDX non-deterministic=output
\begin{ocaml}
# Random.self_init (); Random.int 42;;
0
\end{ocaml}

Check that the command are always executed:

% $MDX non-deterministic=output
\begin{sh}
$ touch hello-world
\end{sh}

\begin{ocaml}
# Sys.file_exists "hello-world";;
- : bool = true
\end{ocaml}

### Non-deterministic Commands

Code blocks with `non-deterministic=command` are never executed unless
`--non-deterministic` is passed as argument to `ocaml-mdx`.

% $MDX non-deterministic=command
\begin{sh}
$ touch toto
\end{sh}

\begin{sh}
$ touch bar
\end{sh}


\begin{ocaml}
# Sys.file_exists "toto";;
- : bool = false
# Sys.file_exists "bar";;
- : bool = true
\end{ocaml}
